\documentclass[12pt, a4paper]{scrartcl}

\usepackage{fullpage}

\author{Richard Polzin}
\title{Robot Soccer Localization in 2016}
\subtitle{Image Based Localization Techniques in Humanoid Robot Soccer League}
\date{\today}

\begin{document}
  \maketitle

  \begin{abstract}
    \textbf{Abstract.}
  \end{abstract}

  \section{Introduction}
  "By the middle of the 21st century, a team of fully autonomous humanoid robot soccer players shall win a soccer game, complying with the official rules of FIFA, against the winner of the most recent World Cup.[]". This is the official goal of the RoboCup initiative. The international scientific initiative has several different leagues and competition domains, where different teams compete. In addition to the competition against the other teams the setting of the challenges become more and more realistic. Autonomous, programmable humanoid robots, called Nao, are used in the KidSize Humanoid Robot Soccer League. A 2016 Nao robot has a height of 58 cm and weighs 4.3 kg. It features an 1.6 GHz Intel Atom processor and 1 GB of RAM. The two legged robot is equipped with multiple sensors, as well as wifi, ethernet and usb.
  In an attempt to make the environment of the game more realistic the characteristics of the soccer field are changed regulary. For examples navigation beacons have been used to flag points of interest on the field. Those beacons were last used in the 2008 RoboCup. Furthermore the field is no longer surrounded by walls and it's size wes increased to 60cm x 90cm. In the 2016 tournament the difficulty was increased by removing the color coding of the team's goals. Up to this year it was possible to differentiate between the teams by those colors. Now the goals are both colored white.
  As the RoboCup competitors constanly try to keep up with the increased difficulty of the game, different self localization techniques have been investigated and researched. Self localization refers to the task of autonomously tracking and maintaining an estimate of a robot's location. It is important in many mobile robot tasks and several different approaches to achieve an good estimate exist. Some rely on ultrasonic sensors, some work with an electrical compass. The Nao robots are built to mimic a human they don't include ultrasonic sensors, but focus on cameras. Therefor this paper will focus on image based techniques.

  \section{Image Based Localization}
  Different Options
  Why Image Based
  Simple Idea
  History of Image Based Approaches

  \section{Challenges}
  - Sensors : Noise, Angle, occlusion
  - Dealing with different lighting situations
  - Being Color invariant
  - Computational Power
  - Real Time
  - Goal: Being able to play outside

  \section{Overview of Techniques}
  - Detect different things
  - Shape detection
  - Line detection
  - Edge detection
  - How to detect them
  - Global / Local Mapping

  \section{Conclusion}


\end{document}
