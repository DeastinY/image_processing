\documentclass[12pt, a4paper]{scrartcl}

\usepackage{fullpage}

\author{Richard Polzin}
\title{Robot Soccer Localization in 2016}
\subtitle{Image Based Localization Techniques in RoboCup Standard Platform League}
\date{\today}

\begin{document}
  \maketitle

  \begin{abstract}
    \textbf{Abstract.}
  \end{abstract}

  \section{Introduction}
  "By the middle of the 21st century, a team of fully autonomous humanoid robot soccer players shall win a soccer game, complying with the official rules of FIFA, against the winner of the most recent World Cup.[]". This is the official goal of the RoboCup initiative. The international scientific initiative has several different leagues and competition domains, where different teams compete. In addition to the competition against the other teams the setting of the challenges become more and more realistic. Autonomous, programmable humanoid robots, called Nao, are used in the RoboCup Standard Platform League (SPL). A Nao V5 robot has a height of 58 cm and weighs 4.3 kg. It features an 1.6 GHz Intel Atom processor and 1 GB of RAM. The two legged robot is equipped with multiple sensors, as well as wifi, ethernet and usb.

  In an attempt to make the environment of the game more realistic the characteristics of the soccer field are changed regulary. For examples navigation beacons have been used to flag points of interest on the field. Those beacons were last used in the 2008 RoboCup. Furthermore the field is no longer surrounded by walls and it's size wes increased to 60cm x 90cm. In the 2015 tournament the difficulty was increased by removing the color coding of the team's goals. Up to this year it was possible to differentiate between the teams by those colors. Now the goals are both colored white.

  As the RoboCup competitors constanly try to keep up with the increased difficulty of the game, different localization techniques have been investigated and researched. Localization refers to the task of autonomously tracking and maintaining an estimate of a robot's location. It is important in many mobile robot tasks and several different approaches to achieve an good estimate exist. Some robots rely on ultrasonic sensors or lasers, some work with an electrical compass. The Nao robots include ultrasonic sensors, as well as cameras. This paper will focus on image based techniques.

  In first section of this paper [] the importance of image based localization is emphasized. Furthermore the concept of those techniques is explained. The second section [] shows the obstacles that have to be overcome to provide a good localization. Section [] describes the techniques used by different Nao teams and explains the important features of their approaches. Finally in section [] the techniques are compared and common features and differences are evaluated.

  \section{Image Based Localization}
  In a game of soccer, where the robots need to react fast and work as a team, localization has an important role. Passing the ball, good positioning and dribbling all require a fast and accurate localization. Localization can be split in local and global localization. Local localization focusses on object's distances relative to the robot. How far are other players, goals, or walls away ? What is the distance to the wall ? Answering those questions already allows for good play and are very important. Global localization on the other hand comes into play when a robot has to be removed from the field or is replaced. Maintaining the global position on the field allows for better positioning and general orientation. Especially with the color coding of the goals being removed, it is important to keep track of the sites of the field. Having great dribbling and shooting many goals is not sufficient if the robot mixes up the enemy goal and it's own.

  Localization of the Nao robot is only possible with the sensors it provides. It offers microphones, tactile sensors, bumbers, four sonar sensors and two cameras. Neither microphones, tactile sensors nor bumpers are valuable to detect objects far away. Only the camera and sonar sensors can be used. The sonar sensors are divided in two receiving and two emitting devices. Detection is possible within 20 cm to 80 cm. With the V4 Version of the Nao the detection range is 20 cm to 2.55 m. The camera sensors are positioned in the head and provide an image with 1280 x 960 pixels at 30 frames per second []. Image [] shows the orientation of the cameras. While the sonar sensors can certainly be used for collision avoidance and player / ball detection, the camera sensors are more often used for localization. This paper, as well as the different techniques explained later on, will not use the sonar senors, but only work with the cameras.

  After concluding that localization is important and the cameras are the most promising sensors to use, now the concept of image based localization is explained. A good self-localization should be fast and reliable. Localization should run fast enough to react quickly to changes in the play and still leave enoungh computational power to the other important algorithms. Furthermore there should always be an solid estimate of the robots position and it should be able to overcome the challenges described in detail in the next section. In general such a localization process consists of three steps. First an image is recorded and some arbitrary feature is detected. For example the image is scanned for the rectangle of a goal. The second step creates an abstract model of the detected feature. A goal is perceived no longer as a flat rectangle, but as an 3D object. It's 3D position is determined and the distance to the robot is calculated. In the final step the robots local map is positioned into the coordinate system of the whole soccer field. Based on the current vision of the robot and for example it's distance to the goal the robot's position on the soccer field is determined. To achieve a fast and reliable localization those three steps are important on their own, as well as their synergy. Most papers publish in the topic of RoboCup SPL localization present localization technique that includes all three steps and adjusts them accordingly to the followed strategy. Due to the limited scope of this paper, the comparison of techniques will focus on the first step : the extraction of important features from the image.

  \section{Challenges}
  Image based Localization in RoboCup SPL has to overcome several obstacles that are important in real life robot applications as well. While the robot's sensors are gradually getting better, the challenges they have to face increases as well. Mimicing a human's vision the cameras on the Nao robots only offer a limited field of view. They have to move their head to fully assess their environment and can't rely on powerfull omnidirectional sensors. Furthermore the cameras have to deal with different lighting conditions. While initially high quality light was required by the rules, nowadays the game isn't interrupted, even when a lightbulb breaks down. Furthermore the initially mentioned color codings were removed and photographers are allowed to freely use their flash.

  All these different aspects can lead to low quality pictures, that has to deal with. In addition to that the already mentioned hardware limitations come into play. While for example shooting in RAW with some high quality DSLR and post processing the image would produce much better results, hardware shall not be changed. The 1.6 GHz of the Atom processor and the 1 GB of RAM are all the resources available. The computations still have to be executed quickly and leave processing time to the other tasks. Ignoring the image quality there are still some challanges to face when it comes to localization. Robots can stand in each others way and occlude goals, the ball or other marks. Relying on color is dangerous as well, as the environment could become more noisy in the future. As the RoboCup initiative aims to win an official FIFA game colorful advertisements, noisy chanting and dramatic lighting make it hard to identify features properly.

  In conclusion there are several different issues that can arise in image based localization. The image can be of a low quality, the features used for localization could be obscured, or the playing environment could change drastically. Those are only the issues the robot has to deal with when it comes to feature extraction and image processing. There are several other challenges in the other part of the localization and there will certainly more, as the game is made more and more realistic.

  \section{Overview of Techniques}
  Image [] shows everything that defines the 2015 RoboCub SPL soccer field. It's mostly white field lines and goals on a green carpet. The goals posts and top cross bar are white, while the net and support structure can be white, gray or black. Lighting can only come from the ceiling and any changes in the lighting will not be cause for delay. Fields can be in visible range of each other, but if they are, they are at least three meters apart.

  To offer information on currently used and successfull techniques the work of the best three teams of the RoboCup 2016 in Leipzig, Germany will be considered. The winning teams are B-Human, UT Austin Villa and Nao-Team HTWK, in descending order. In the description for RoboCup 2016 each of these teams shares their publications since RoboCup 2015.

  - Detect different things
  - Shape detection
  - Line detection
  - Edge detection
  - How to detect them
  - Global / Local Mapping

  \section{Conclusion}


\end{document}
